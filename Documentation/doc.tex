\documentclass[12pt,a4paper]{article}
\usepackage[polish]{babel}
\usepackage[T1]{fontenc}
\usepackage{lmodern}
\usepackage[utf8x]{inputenc}
\usepackage{hyperref}
\usepackage{url}
\usepackage{graphicx}
\usepackage{listings}
\title{Landscape Generator\\Inżyniera Oprogramowania}
\author{Artur Bednarczyk, Dawid Grajewski, Tomasz Januszek\\Politechnika Śląska\\Wydział Matematyki Stosowanej\\Informatyka, semestr V}
\date{\today}

\begin{document}
\maketitle
\newpage
\tableofcontents
\newpage
\section{O projekcie}
\subsection{Zespół}
Artur Bednarczyk, Dawid Grajewski, Tomasz Januszek.
\subsection{Temat}
\paragraph{Generowanie realistycznych krajobrazów 3D}
Generowanie w języku wysokiego poziomu (nie w generatorach typu Unity) losowych krajobrazów z uwzględnieniem zadanych parametrów: stromizny terenu, poziomu wody, kolorów na danej wysokości lub obszarzem wizualizacja i symulacja przemieszczania kamery.

Bonus: dodanie roślinności (drzewa, krzewy - co najmniej 3 rodzaje) o zadanej częstości i miejscu występowania.

\section{Projekt}
\subsection{Plany i pomysły}

\subsection{UI/UX}
\subsubsection{Zawartość}
UI generatora pozwoli na ustalenie parametrów zgodnie z którymi zostanie wygenerowany krajobraz. Parametry te to: \begin{itemize}
\item Stromizna terenu
\item Poziom wody
\item Górzystość
\item Gęstość dodatkowych elementów
\item Położenie dodatkowych elementów
Dodatkowo użytkownik uzyska możliwość zapisywania wygenerowanego terenu 
\end{itemize}
\subsubsection{Projekty UI}
includegraphics
\section{Teoria}
\subsection{Losowość}
Celem projektu jest wygenerowanie losowego krajobrazu, więc sama losowość jest bardzo ważna. Każdy kolejny krajobraz powinien być inny, a prawdopodobieństwo wystąpienia dwóch podobnych obrazów bardzo niskie. Takie efekty możemy uzyskać dzięki algorytmowi jakim jest "Szum Perlina"
\subsection{Algorytmy}
\subsubsection{Szum Perlina}
Bazę dla naszego sposobu generowania danych potrzebnych do wyświetlenia zróżnicowanego terenu stanowi szum Perlina. Jest to jeden z typów szumu gradientowego utworzony przez Kena Perlina już w 1983 roku dla potrzeby tworzenia realistycznych grafik komputerowych. 
Algorytm składa się z trzech kroków
\begin{itemize}
\item Pierwszym krokiem jest zdefiniowanie wielowymiarowej siatki jednostkowych wektorów rozpatrywanego gradientu. W naszym przypadku są to wartości losowe z zakresu (0, 1) liczb rzeczywistych. Dla jednowymiarowego przypadku byłyby dostępne jedynie wartości -1 albo 1
\item Kolejno iteruje się po podawanych punktach. Punkt wpada do pewnej komórki wygenerowanej siatki. Następnie wyliczany jest iloczyn skalarny między punktem a wektorem każdego z rogów komórki (a więc ich odległość), po czym zapisane zostają w pamięci.
\item Przeprowadzona zostaje interpolacja dla każdej pary punktów z uwzględnieniem funkcji wygładzającej.
\end{itemize}
Wynikowo otrzymujemy wielowymiarową macierz (lub tensor) zawierający wartości zamknięte w pewnych granicach. Jest możliwe nakładanie na siebie wielu takich macierzy generowanych dla różnych częstotliwości siatki w celu uzyskania różnych ułożeń lub skupień wartości.
\section{Narzędzia}
\subsection{Kontrola wersji}
Do zarządzania kodem i wersjami projektu wykorzystujemy narzędzie Git. Korzystamy z platformy GitHub jako repozytorium dostępnego online. Dobór narzędzi służących do korzystania z repozytorium to sprawa indywidualna każdego członka zespołu, ponieważ nie ma ona wpływu na sam projekt.
\subsection{Zarzadzanie zespołem}
Trello - Kanban Board - to tutaj rozpisujemy zadania i przydzielamy je sobie, określamy również terminy i planujemy.
\subsection{Środowisko}
Visual Studio 2015 \\
Do wyświetlenia terenu wykorzystujemy zestaw funkcji API wspomagający generowanie grafiki – DirectX wraz z DirectX Software Development Kit używany z C\#.
\section{Aplikacja}
\subsection{Architektura}
Utworzony silnik graficzny jest wzorowany na istniejących już popularnych silnikach do gier takich jak Unity, Cry czy Unreal Engine. Opiera się on na trzech głównych składowych: \begin{itemize}
\item scena - obiekt nadrzędny, to w nim rozgrywa się akcja.
\item encja - umieszczona w scenie. Jest to pojedynczy agent w scenie, np. kamera, teren czy postać. 
\item komponent - to składowa encji - odpowiada za część logiki encji (przykładowo, encja "player" może mieć komponenty takie jak "moveComponent" czy "shootComponent"
\\
Silnik jest zbudowany zgodnie z wzrocem kompozytu - encja składa się z kompenentów, ale pojedynczy komponent może zawierać kolejne, zagnieżdżone komponenty. Pozwala to na rekurencyjne wykonywanie metod w obiektach o tym samym interfejsie.
\end{itemize}
\subsection{Struktury danych}
Dane o wygenerowanym terenie będą przechowywane w pliku .json. Umożliwi nam to zapisanie i późniejsze odtworzenie terenu.
\subsection{Schemat graficzny struktury systemu}
\includegraphics[width=1\textwidth]{images/klasy.png}
\subsection{Komunikacja między modułami}
IN PROGRESS
\section{API}
\subsection{Perlin}
Tworzymy obiekt PerlinNoise, który wymaga podania rozmiaru i wymiarów.
Pobieranie wartości za pomocą metody:\\
CalculatePerlin(x,y);
\end{document}